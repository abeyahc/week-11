\documentclass{report}

\input{preamble}
\input{macros}
\input{letterfonts}

\usepackage{mathtools}

\title{\Huge{Discrete Mathematics}\\Week 11}
\author{\huge{Abeyah Calpatura}}
\date{}

\begin{document}
\maketitle
\section*{9.1}
\subsection*{Exercises} \\
\text{Abeyah Calpatura} \\
\#4, 7, 11, 19 \\

\noindent \textbf{\#4}
\sol{ The event that the chosen card is black and has an even number on it. 
\begin{align*}
        & \text{Black cards consist of spades and clubs, which are 26 cards in total which is the number of possible outcomes} \\
        & \text{Even numbers consist of 2, 4, 6, 8, 10 meaning 5 favorable outcomes} \\
        & \text{$P(E) = \: \frac{\text{# of favorable outcomes}}{\text{# of possible outcomes}}  = \frac{5}{26}  = 19.23\%$}
\end{align*}
}

\noindent \textbf{\#7}
\sol{ The even that the sum of numbers showing face up is 8. 
\vspace{4cm}
    \begin{align*}
        & \text{$P(E) = \frac{\text{# of favorable outcomes}}{\text{# of possible outcomes}} = \frac{5}{36} = 13.89\% $}
    \end{align*}
}

\noindent \textbf{\#11} \\
Suppose that a coin is tossed three times and the side showing face up on each toss is noted. Suppose also that on each toss heads and tails are equally likely. Let HHT indicate the outcome heads on the first two tosses and tails on the third, THT the outcome tails on the first and third tosses and heads on the second, and so forth. \\
\textbf{a.}
\sol{ Eight elements in the sample space whose outcomes are all the possible head-tail sequences obtained sequences obtained in the three tosses.
\begin{align*}
    & S = \{HHH, HHT, HTH, HTT, THH, THT, TTH, TTT\} \\
    & \text{The number of elements in the sample space is 8.}
\end{align*}
}
\textbf{bi.}
\sol{ The event that exactly one toss results in a head.
\begin{align*}
    & E = \{HTT, THT, TTH\} \\
    & \text{The number of elements in the event is 3.} \\
    & \text{The probability of the event is $P(E) = \frac{3}{8} = 37.5\%$} \\
\end{align*}
}
\textbf{bii.}
\sol{ The event that at least two tosses result in a head.
\begin{align*}
    & E = \{HHT, HTH, THH, HHH\} \\
    & \text{The number of elements in the event is 4.} \\
    & \text{The probability of the event is $P(E) = \frac{4}{8} = 50\%$} \\
\end{align*}
}
\textbf{biii.}
\sol{ The event that no head is obtained.
\begin{align*}
    & E = \{TTT\} \\
    & \text{The number of elements in the event is 1.} \\
    & \text{The probability of the event is $P(E) = \frac{1}{8} = 12.5\%$} \\
\end{align*}
}

\noindent \textbf{\#19}
An urn contains two blue balls (denoted as $B_1$ and $B_2$) and three white balls (denoted $W_1$, $W_2$, and $W_3$). One ball is drawn, its color is recorded, and it is replaced in the urn. Then another ball is drawn and its color is recorded. \\
\textbf{a.}
\sol{ List all possible 25 possible outcomes of the experiment.
\begin{align*}
    & S = \{B_1B_1, B_1B_2, B_2B_1, B_1W_1, W_1B_1, B_1W_2, W_2B_1, B_1W_3, W_3B_1, B_2B_2, B_2W_1, W_1B_2, B_2W_2, \\
    & W_2B_2, B_2W_3, W_3B_2, W_1W_1, W_1W_2, W_2W_1, W_1W_3, W_3W_1, W_2W_2, W_2W_3, W_3W_2, W_3W_3\} \\
    & \text{The number of elements in the sample space is 25.}
\end{align*} 
}
\textbf{b.}
\sol{Consider the event that the first ball that is drawn is blue. List all the outcomes in the event. What is the probability of the event?
\begin{align*}
    & \text{S = $\{(B_1B_1), (B_1B_2), (B_2B_1), (B_2B_2), (B_1W_1), (B_1W_2), (B_1W_3), (B_2W_1), (B_2W_2), (B_2W_3)\}$} \\
    & \text{The number of elements in the event is 10.} \\
    & \text{The probability of the event is $P(E) = \frac{10}{25} = 40\%$} 
\end{align*}
}  
\textbf{c.}
\sol{ Consider the event that only white balls are drawn. List all the outcomes in the event. What is the probability of the event?
\begin{align*}
    & \text{S = $\{(W_1W_1), (W_1W_2), (W_1W_3), (W_2W_1), (W_2W_2), (W_2W_3), (W_3W_1), (W_3W_2), (W_3W_3)\}$} \\
    & \text{The number of elements in the event is 9.} \\
    & \text{The probability of the event is $P(E) = \frac{9}{25} = 36\%$} 
\end{align*}
}  

\newpage
\section*{9.2}
\subsection*{Exercises} \\
\text{Abeyah Calpatura} \\
\#5, 7, 15, 17, 32, 36, 39bd \\

\noindent \textbf{\#5}
\sol{ In a competition between players X and Y, the first player to win three games in a row or a total of four games wins. How many ways can the competition be played if X wins the first game and Y wins the second and third games?
    \vspace{5cm}
    \begin{align*}
        & \text{The number of ways the competition can be played is 7.}
    \end{align*}
} 

\noindent \textbf{\#7}
One urn contains one blue ball ($B_1$) and three red balls ($R_1, R_2, R_3$). A second urn contains two red balls ($R_4, R_5$), and two blue balls ($B_2, B_3$). An experiment is performed in which one of the two urns is chosen at random and then two balls are randomly chosen from it, on after the other without replacement. \\
\textbf{a.}
\sol{ Construct the possibility tree showing all possible outcomes of this experiment. 
\vspace{5cm}
} \\
\textbf{b.}
\sol{ What is the total number of outcomes of this experiment?
\begin{align*}
    & \text{There are 24 outcomes of this experiment.} 
\end{align*}
} 
\textbf{c.}
\sol{
    \begin{align*}
        & \text{8 of the 24 outcomes contain two red balls.} 
        & \text{The probability of drawing two red balls is $P(E) = \frac{8}{24} = 33.33\%$}
    \end{align*}
}  

\noindent \textbf{\#15}
A combination lock requires three selections of numbers, each from 1 through 30. 
\textbf{a.}
\sol{ How many different combinations are possible?
\begin{align*}
    & \text{The number of different combinations possible is $30^3 = 27,000$}
\end{align*}
} 
\textbf{b.}
\sol{ Supposethe locks are constructed in such a way that no number may be used twice. How many different combinations are possible?
\begin{align*}
    & \text{The number of different combinations possible is $30 \cdot 29 \cdot 28 = 24,360$}
\end{align*}
} 

\noindent \textbf{\#17a.} 
\sol{ How many integers are there from 1000 through 9999?
\begin{align*}
    & \text{10 possible digits (0,1,2,3,4,5,6,7,8,9)} 
    & \text{The number of integers from 1000 through 9999 is $9 \cdot 10 \cdot 10 \cdot 10 = 9000$}
\end{align*}
} 
\noindent \textbf{\#17b.}
\sol{ How many odd integers are there from 1000 through 9999?
\begin{align*}
    & \text{10 possible digits (0,1,2,3,4,5,6,7,8,9)} \\
    & \text{There are 5 odd digits (1,3,5,7,9)} \\
    & \text{Using multiplication rule, the number of odd integers from 1000 through 9999 is $9 \cdot 10 \cdot 10 \cdot 5 = 4500$ }
\end{align*}
}  
\noindent \textbf{#17c.}
\sol{ How many integers from 1000 through 9999 have distinct digits?
\begin{align*}
    & \text{10 possible digits (0,1,2,3,4,5,6,7,8,9)} \\
    & \text{The number of integers from 1000 through 9999 with distinct digits is $9 \cdot 9 \cdot 8 \cdot 7 = 4536$}
\end{align*}
}  
\noindent \textbf{#17d.}
\sol{ How many odd integers from 1000 through 9999 have distinct digits?
\begin{align*}
    & \text{10 possible digits (0,1,2,3,4,5,6,7,8,9)} \\
    & \text{There are 5 odd digits (1,3,5,7,9)} \\
    & \text{The number of odd integers from 1000 through 9999 with distinct digits is $5 \cdot 8 \cdot 8 \cdot 7 = 2240$}
\end{align*}
} 
\noindent \textbf{\#17e.}
\sol{Probability that a randomly chosen four-digit integer has distinct digits? Has distinct digits and is odd?
\begin{align*}
    & \text{The probability that a randomly chosen four-digit integer has distinct digits is $\frac{4536}{9000} = 50.4\%$} \\
    & \text{The probability that a randomly chosen four-digit integer has distinct digits and is odd is $\frac{2240}{9000} = 24.89\%$}
\end{align*}
} 

\noindent \textbf{\#32a.}
\sol{ How many ways can the letters of the owrd ALGORITHM be arranged in a row?
\begin{align*}
    & \text{Word contains 9 elements, $9! = 9 \cdot 8 \cdot 7 \cdot 6 \cdot 5 \cdot 4 \cdot 3 \cdot 2 \cdot 1 = 362,880$}
\end{align*}
} 
\noindent \textbf{\#32b.}
\sol{ Arranged in a row if AL remained together in ALGORITHM?
\begin{align*}
    & \text{\{AL, G, O, R, I, T, H, M\}} \\
    & \text{Word contains 8 elements, $8! = 8 \cdot 7 \cdot 6 \cdot 5 \cdot 4 \cdot 3 \cdot 2 \cdot 1 = 40,320$} 
\end{align*}
} 
\noindent \textbf{\#32c.}
\sol{ Arranged in a row if GOR remained together in ALGORITHM?
\begin{align*}
    & \text{\{A, L, GOR, I, T, H, M\}} \\
    & \text{Word contains 7 elements, $7! = 7 \cdot 6 \cdot 5 \cdot 4 \cdot 3 \cdot 2 \cdot 1 = 5,040$}
\end{align*}
}

\noindent \textbf{\#36}
\sol{Write all the 3-permutations of \{s, t, u, v\}
\begin{align*}
    & \text{The 3-permutations of \{s, t, u, v\} are \{stu, stv, sut, suv, svu, svv, tsu, tsv, tus, tuv, tvu, tvv, } \\
    & \text{ust, usv, uts, utv, uvt, uvv, vst, vsu, vts, vtu, vut, vuv\}} \\ \\ 
    & \text{$P(4, 3) = \frac{4!}{(4-3)!} = \frac{4 \cdot 3 \cdot 2 \cdot 1}{1} = 24 $ } 
\end{align*}
} 

\noindent \textbf{\#39b.}
\sol{How many ways can six of the letters of the word ALGORITHM be selected and written in a row?
\begin{align*}
    & \text{ $n = 9$ and $r = 6$ } \\
    & \text{$P(9,6) = \frac{9! }{(9-6)!} = \frac{9!}{3!} = 60,480  $} 
\end{align*}
} 
\noindent \textbf{\#39d.}
\sol{ How many ways can six of the letters of the word ALGORITHM be selected and written in a row if the first two letters must be OR?
\begin{align*}
    & \text{4 out of the 7 letters}
    & \text{$P(7,4) = \frac{7!}{(7-4)!} = \frac{7!}{3!} = 840$} 
\end{align*}
} 



\end{document}